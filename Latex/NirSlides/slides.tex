\documentclass[ucs, notheorems, handout]{beamer}

\usetheme[numbers,totalnumbers,compress, nologo]{Statmod}
\usefonttheme[onlymath]{serif}
\setbeamertemplate{navigation symbols}{}

\mode<handout> {
    \usepackage{pgfpages}
    \setbeameroption{show notes}
    \pgfpagesuselayout{2 on 1}[a4paper, border shrink=5mm]
    \setbeamercolor{note page}{bg=white}
    \setbeamercolor{note title}{bg=gray!10}
    \setbeamercolor{note date}{fg=gray!10}
}

\usepackage[utf8x]{inputenc}
\usepackage[T2A]{fontenc}
\usepackage[russian]{babel}
\usepackage{tikz}
\usepackage{ragged2e}

\newtheorem{theorem}{Теорема}

\newcommand{\rank}{\mathsf{rank}\ }
\newcommand{\Lrank}{\mathsf{rank}_L\ }
\newcommand{\T}{\mathcal{T}}
\newcommand{\F}{\mathsf{F}}
\newcommand{\MF}{\vec{\F}}
\newcommand{\sfS}{\mathsf{S}}
\newcommand{\sfR}{\mathsf{R}}
\newcommand{\MS}{\vec{\sfS}}
\newcommand{\MSE}{\mathsf{MSE}}
\newcommand{\SSA}{\mathsf{SSA}}
\newcommand{\MSSA}{\mathsf{MSSA}}
\newcommand{\ProjSSA}{\mathsf{ProjSSA}}
\newcommand{\mean}{\mathsf{mean}}
\newcommand{\X}{\mathbf{X}}
\newcommand{\wX}{\overset{\wedge}{\X}}


\title[Поддерживающие ряды MSSA]{Исследование условий для поддерживающих временных рядов в MSSA}

\author{Ткаченко Егор Андреевич, гр.19.Б04-мм}

\institute[Санкт-Петербургский Государственный Университет]{%
    \small
    Санкт-Петербургский государственный университет\\
    Прикладная математика и информатика\\
    Вычислительная стохастика и статистические модели\\
    \vspace{1.25cm}
    Отчет по производственной практике (семестр 6)}

\date[Зачет]{Санкт-Петербург, 2022}

\subject{Talks}

\begin{document}

\begin{frame}[plain]
    \titlepage

    \note{Научный руководитель  к.ф.-м.н., доцент Голяндина Нина Эдуардовна,\\
    кафедра статистического моделирования}
\end{frame}


%\section{Короткая тема}
%\subsection{Общие слова}

\setbeameroption{show notes}

\begin{frame}{Введение}
    Тут какое-то введение.
    Что за задача решается, какое метод используется, какая цель работы.

    \note{
        Полезность умения строить прогнозы не нуждается в доказательстве. Прогноз временных рядов может использоваться в прогнозе погоды, приливов, спроса на товары и многом другом.
         
        С помощью книги \cite{SSA_with_R} был изучен базовый $\SSA$, разложение рядов, заполнение пропусков в данных, прогноз и базовый $\MSSA$. Для работы с временными рядами и их прогнозом использовался пакет Rssa. Проведены эксперименты с простейшими моделями сигналов для изучения связи между согласованностью сигналов и поддерживающими рядами. 
        Исследовано, при каких отклонениях, сигналы с одинаковой структурой перестают быть согласованными. Проведено сравнение линейных рядов и их аппроксимаций экспонентой как поддерживающих.
    }
\end{frame}

\begin{frame}{Обозначения и известные результаты}
    Обозначения, нужные понятия и нужные известные результаты.

    \note{
        Текст про это
    }
\end{frame}

\begin{frame}{Полученные результаты}
    Тут ваши личные результаты

    \note{
        Текст про это
    }
\end{frame}

\begin{frame}{Заключение}
    Тут какое-то заключение.
    Что сделано, резюме по результатам.

    \note{
        Текст про заключение
    }
\end{frame}

\begin{frame}{Список литературы}
    \bibliographystyle{ugost2008}
	\bibliography{references}
% \begin{thebibliography}{3}
% \bibitem{main}Самая главная работ
% \bibitem{another} Еще одна
% \end{thebibliography}    

    \note{
        На данном слайде представлен список основных источников, используемых в моей работе.
    }
\end{frame}

\end{document}
